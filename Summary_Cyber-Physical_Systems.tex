\documentclass[10pt,a4paper]{article}
\usepackage[utf8x]{inputenc}
\usepackage{ucs}
\usepackage[left=2.00cm, right=2.00cm, top=2.00cm, bottom=2.00cm]{geometry}
\renewcommand\familydefault{\sfdefault}

\title{Summary - \lecture}
\author{}
\date{}


%costum layout
\setlength{\parindent}{0cm}
\usepackage{fancyhdr}
\pagestyle{fancy}
\fancyhf{}
\fancyhead[L]{
	\strut\rlap{\colorlayout\rule[-\dp\strutbox]{\headwidth}{\headheight}}
	\textcolor {white} {Summary: \lecture}}
\fancyfoot[L]{
	\strut\rlap{\colorlayout\rule[-\dp\strutbox]{\headwidth}{\headheight}}
	\textcolor {white} {last changed: \today}}
\fancyhead[R]{\textcolor{white}{\semseter}}
\fancyfoot[R]{\textcolor{white} {\thepage}}


%math
\usepackage{amsmath}
\usepackage{amsfonts}
\usepackage{amssymb}
\usepackage{amstext}
\usepackage{mathtools}


%graphics
\usepackage{graphicx}
\usepackage{floatflt}
\usepackage{float}


%tabular
\usepackage{tabularx}
\usepackage[font=small,labelfont=small]{caption}
\usepackage{colortbl}
\usepackage[dvipsnames]{xcolor}
\renewcommand{\arraystretch}{1.5}
%\arrayrulecolor{white}


%tikz
\usepackage{tikz}
\usetikzlibrary{shapes, petri}
\tikzstyle{ell}=[ellipse,draw, yshift=-2mm]
\tikzstyle{rec} = [rectangle, draw]
\tikzstyle{dia} = [diamond, aspect=2, draw, yshift=-5mm]
\tikzstyle{cir} = [circle, draw, minimum size=3mm]
\tikzstyle{arrHV} = [to path={-| (\tikztotarget)}]
\tikzstyle{arrVH} = [to path={|- (\tikztotarget)}]
\tikzstyle{whileright} = [xshift=20mm, yshift=-3mm]
\tikzstyle{whileleft} = [xshift=-20mm, yshift=-3mm]
\tikzstyle{txtright} = [above, xshift=15mm]
\tikzstyle{txtleft} = [above, xshift=-15mm]
\tikzstyle{empty} = [coordinate]
\usetikzlibrary{positioning}


%listings
\usepackage{listings}
\lstdefinestyle{costum} {
	language=Bash,
	basicstyle=\footnotesize\ttfamily,
	keywordstyle=\bfseries\color{cyan!50!blue},
	commentstyle=\itshape\color{black!50},
	%identifierstyle=\color{blue},
	stringstyle=\color{green!50!black},
	morekeywords={returns, loop, each},
	escapeinside={\%*}{*)}
}
\lstset{style=costum}


%%custom title color
%\usepackage{titlesec}
%\setcounter{secnumdepth}{4}
%
%\titleformat{\section}
%{\color{cyan!80!blue}\normalfont\Large\bfseries}
%{\color{black}\thesection}{1em}{}
%
%\titleformat{\subsubsection}
%{\color{blue!30!black!70}\normalfont\bfseries}
%{\color{black}\thesection}{1em}{}
%
%\titleformat{\paragraph}
%{\color{green!30!black!70}\normalfont\normalsize\bfseries}{\theparagraph}{1em}{}
%\titlespacing*{\paragraph}
%{0pt}{3.25ex plus 1ex minus .2ex}{1.5ex plus .2ex}


%tab
\newcommand{\tab}[1][1]{\hspace*{#1cm}}


%hyperref
\usepackage{hyperref}


%vector
\newcommand{\vect}[1]{\ensuremath{\begin{bmatrix}#1\end{bmatrix}}}

%TODO
%config
\newcommand{\lecture}{Cyber-Physical Systems} %title of the lecture
\newcommand{\lecturer}{Althoff M.} %lecturer of the lecture
\newcommand{\semseter}{summer semester 2020} %semester of this lecture, e.g., summer semester 2019
\newcommand{\colorlayout}{\color{red!50!black}} %color of the title bar, see colors

%colors, e.g,
%cyan!50!blue
%green!50!black
%orange!50!black

%user


%TODO
% a user defined todo list


%%Check
%Introduction (y)
%



\begin{document}
\tableofcontents
\pagebreak

\section{Introduction}
\subsection{Typical Sensors and Actuators}
\begin{itemize}
	\item Sensors
	\begin{itemize}
		\item acceleration sensor
		\item light sensor
		\item force sensor
		\item temperature sensor
		\item video camera
		\item pressure sensor
		\item angle sensor
		\item LIDAR
	\end{itemize}
	\item Actuators
	\begin{itemize}
		\item electric motor
		\item hydraulic/pneumatic cylinder
		\item magnetic valve
		\item relay
		\item heating
		\item piezo actuator
		\item pump
		\item laser
	\end{itemize}
\end{itemize}

\subsection{Model-Based Design}
\subsubsection{Process}
\begin{enumerate}
	\item Modeling
	\item Design
	\item Analysis
	\item Deployment
\end{enumerate}

\subsubsection{Advantages}
\begin{itemize}
	\item Improvement of the product quality
	\item Handling complexity
	\item Shorter development times
\end{itemize}

\subsubsection{Concept of Systems}
\begin{itemize}
	\item \textbf{System:} Is a set of interacting or independent components that is distinguished from its environment by a system boundary
	\item \textbf{System boundary:} Describes the exchange of a system with its environment via inputs and outputs
	\item \textbf{Subsystem:} System in a system
\end{itemize}

\subsection{Signal Types}
\subsubsection{Continuous Signals}
$$
	f : \mathbb{R}^+_0 → \mathbb{R}
$$

\subsubsection{Discrete-Time Signal}
$$
	f : \mathcal D → \mathbb{R}
$$ where $\mathcal D$ is a countable set, e.g. $\mathcal D = \{t_1, t_2, \dots \} \text{ or } \mathcal D = \mathbb{N}_0$

\subsubsection{Discrete-Value Signal}
$$
	f : \mathbb{R}^+_0 → \mathcal D
$$ where $\mathcal D$ is a countable set, e.g. $\mathcal D  = \{0, 1\}$

\subsubsection{Discrete-Time and Discrete-Value Signal}
$$
	f : \mathcal D → \tilde{\mathcal D}
$$ where $\mathcal D, \tilde{\mathcal D}$ are countable sets

\subsection{Systems}
\begin{tabularx}{\columnwidth}{l|X|X|X}
	& Discrete-State System & Continuous-State System & Hybrid System \\
	\hline
	\hline
	States/Inputs/Outputs & discrete & continuous & discrete and continuous \\
	Time variable & $t_k$ (discrete) & $t$ (continuous) & $t$ (continuous)\\
	Input variable &  $u(t_k) \in \tilde{\mathcal D}_1$ & $u(t) \in \mathbb R$ & $u(t) \in \tilde{\mathcal D}_1$ or $\mathbb R$ \\
	Output variable &  $y(t_k) \in \tilde{\mathcal D}_2$ & $y(t) \in \mathbb R$ & $y(t) \in \tilde{\mathcal D}_2$ or $\mathbb R$ \\
	State vector & $z(t_k)$ & $x(t)$ & $x(t)$ \\
	Equations & $z(t_{k+1}) = A z(t_k) + b u(t_k)$ & $\dot x(t) = A x(t) + b u(t)$ & $\dot x(t) = A x(t) + b u(t)$ \\
	& $y(t_k) = c^T z(t_k)$ & $y(t) = c^T x(t)$ & $y(t) = c^T x(t)$
	
\end{tabularx}

\subsubsection{Properties}


\textbf{State of a dynamic system}
\begin{itemize}
	\item A state vector $x$ consists of the (smallest) number of variables that need to be specified at the initial time $t_0$ so that the future behavior is uniquely defined for a given input signal $u(t)$
\end{itemize}

\textbf{Static and Dynamic Systems}
\begin{itemize}
	\item \textbf{Static System:} No state required
	\item \textbf{Dynamic System:} State required
\end{itemize}

\textbf{Time-Invariant and Time-Variant Systems}
\begin{itemize}
	\item \textbf{Time-invariant system:} Shift of time does not change the outcome
	\item \textbf{Time-variant system:} Shift of time alters the outcome
\end{itemize}

\textbf{Deterministic and Non-Deterministic Systems}
\begin{itemize}
	\item \textbf{Deterministic system:} For an initial state $x(0)$ and a given input signal $u(t)$, there exists a unique solution of the state and the output
	\item \textbf{Non-deterministic system:} The evolution of the state and the output is not uniquely determined by the initial state and the input signal
\end{itemize}

\textbf{Causal, Acausal, and anticausal Systems}
\begin{itemize}
	\item \textbf{Causal sytem:} The output only depends on past and current inputs
	\item \textbf{Acausal system:} The output also depends on future inputs
	\item \textbf{Anticausal system:} The output only depends on future inputs
\end{itemize}

\pagebreak
\section*{Notes}
This is a summary of the lecture~\lecture~of the Technical University Munich.
This lecture was presented by~\lecturer~in the~\semseter.
This summary was created by Gaida B.
All provided information is without guarantee.


%\section*{References}
%author. \textit{title}. publisher. location, year, 

\end{document}

%TODO check for todos








